\documentclass{beamer}

% Must be loaded first
\usepackage{tikz}

\usepackage[utf8]{inputenc}
\usepackage{textpos}

% Font configuration
\usepackage{fontspec}

\input{font.tex}

% Tikz for beautiful drawings
\usetikzlibrary{mindmap,backgrounds}
\usetikzlibrary{arrows.meta,arrows}
\usetikzlibrary{shapes.geometric}

% Minted configuration for source code highlighting
\usepackage{minted}
\setminted{highlightcolor=black!5, linenos}
\setminted{style=perldoc}

\usepackage[listings, minted]{tcolorbox}
\tcbset{left=6mm}

% Use the include theme
\usetheme{codecentric}

% Metadata
\title{How This Presentation Was Made}
\author{Markus Hauck @markus1189}

\newcommand{\recipe}{%
  \begin{itemize}
  \item AST
  \item \texttt{inject}
  \item interpreter
  \item check laws
  \end{itemize}
}

% The presentation content
\begin{document}

\begin{frame}[noframenumbering,plain]
  \titlepage{}
\end{frame}

\section{Introduction}\label{sec:introduction}

\begin{frame}
  \frametitle{Presentations}
  \begin{center}
    \includegraphics[width=\textwidth]{ditaa/presentations.png}
  \end{center}
\end{frame}

\begin{frame}
  \frametitle{Presentations: But How}
  \begin{itemize}
  \item powerpoint/keynote/google slides/\ldots{}
  \item but you can't use \texttt{git}
  \item requirement: some markup language-ish thing
  \item e.g., pandoc / LaTeX
  \item text handled, what about pictures/code/etc?
  \end{itemize}
\end{frame}

\begin{frame}
  \frametitle{How It All Started}
  \begin{itemize}
  \item Me writing presentation be like:
  \item fighting graphical editor more than focused on content
  \item logical step: switch to something that is text based
  \item how to handle generated pictures
  \item how to handle code
  \end{itemize}
\end{frame}

\begin{frame}
  \frametitle{Writing Presentation}
  \begin{itemize}
  \item quick and dirty: google slides, powerpoint, keynote
  \item can't use version control like git
  \item paste pictures
  \item want to change something? awful
  \end{itemize}
\end{frame}

\begin{frame}
  \frametitle{Used Tools}
  \begin{itemize}
  \item Nix for system dependencies + build env
  \item Shake to write a custom build system
  \item Dhall for easy configuration
  \item LaTeX for slides
  \item ditaa, graphviz
  \item more\ldots
  \end{itemize}
\end{frame}

\begin{frame}
  \frametitle{Presentations}
  \begin{itemize}
  \item having CI would be nice!
  \item ``just'' check out from github and build it?
  \item good luck with that!
  \end{itemize}
\end{frame}

\begin{frame}
  \frametitle{Pain Points}
  \begin{itemize}
  \item you have to setup LaTeX
  \item plus all required packages (most of the time: \textbf{all})
  \item hope that everything is up to date and try to build it
  \item notice missing fonts, try to find them somewhere
  \item notice missing python package (pygments), install it
  \item lots of trial and error
  \end{itemize}
\end{frame}

\begin{frame}
  \frametitle{Presentation As Code}
  \begin{itemize}
  \item reproducible: same description for CI and local machine
  \item single step: one command
  \item declarative: generate from description
  \item checked: source code compiles
  \end{itemize}
\end{frame}

\begin{frame}
  \frametitle{Scenarios}
  \begin{itemize}
  \item Nix, Shake and Dhall
  \item Scenario 1: Dependencies
  \item Scenario 2a: Editing Code
  \item Scenario 2b: Checking Snippets
  \item Scenario 3: Generated Pictures
  \item Scenario 4: Continuous Integration
  \end{itemize}
\end{frame}

\section{Shake}

\begin{frame}
  \frametitle{Shake}
  \begin{itemize}
  \item \url{shakebuild.com/manual}
  \item Shake is a Haskell \textbf{library} for writing build systems
  \item \texttt{Shake} vs \texttt{make} is like \texttt{Monad} vs \texttt{Applicative}
  \item integrates well with other libraries and system tools
  \item the backbone of this presentation
  \end{itemize}
\end{frame}

\begin{frame}
  \frametitle{Shake}
  \begin{itemize}
  \item specify rules to create output from some input
  \item avoid rebuilds of unchanged things
  \item ``just'' a library, rest is up to you
  \end{itemize}
\end{frame}

\begin{frame}[fragile]
  \frametitle{Shake Rules}
  \begin{minted}{haskell}
--  +---------------- file pattern to match
--  |
--  |       +-------- target path to create
--  |       |
--  v       v
pattern %> \out -> do
  action1          -- <--\
  action2          -- <---+- Actions to build 'out'
  action3          -- <--/
  \end{minted}
\end{frame}

\begin{frame}[fragile]
  \frametitle{Shake Rules}
  \begin{minted}{haskell}
"*.txt" %> \out -> do
  putNormal "Debug"
  cmd "touch" [out]
  \end{minted}
\end{frame}

\begin{frame}[fragile]
  \frametitle{Shake Rules}
  \inputminted[autogobble]{haskell}{snippets/pdf-rule.hs}
\end{frame}

\begin{frame}
  \frametitle{Shake Rules}
  \begin{center}
    \includegraphics[width=0.8\textwidth]{graphviz/rules.png}
  \end{center}
\end{frame}

\begin{frame}
  \frametitle{Editing Code}
  \begin{itemize}
  \item Step 1: Implement your code in a Scala project
  \item Step 2: Wild Copy And Paste Into Presentation
  \item Step 3: Reformat To Fit Slide
  \item Step 4: Change Original Source Code
  \item Step 5: Wild Editing Of Code on Slides
  \item Step 6: Being suspicious that something is broken (optional)
  \end{itemize}
\end{frame}

\begin{frame}[fragile]
  \frametitle{Extract Code}
  \begin{itemize}
  \item common: extract based on lines
  \item after edit / formatting / \ldots they change
  \item not what we want
  \end{itemize}
\end{frame}

\begin{frame}
  \frametitle{Editing Code}
  \begin{itemize}
  \item idea: extract source code directly from actual project
  \item use comments to delimit ``snippets''
  \item write code to extract everything in between
  \end{itemize}
\end{frame}

\begin{frame}
  \frametitle{Editing Code}
  \begin{itemize}
  \item add comments in the code
  \item write a small ``snippet'' file
  \item let shake automatically extract snippets
  \item include code snippets in presentation
  \end{itemize}
\end{frame}

\begin{frame}
  \frametitle{Annotating Code for Snippets (META)}
  \begin{center}
    \inputminted[autogobble]{haskell}{snippets/outer-pdf-rule.hs}
  \end{center}
\end{frame}

\begin{frame}
  \frametitle{Snippet File}
  \begin{itemize}
  \item for snippet files we use Dhall
  \item non-turing complete programming language for configuration
  \end{itemize}
  \begin{center}
    \inputminted{text}{snippets/pdf-rule.snippet}
  \end{center}
\end{frame}

\begin{frame}
  \frametitle{Snippet Rule \textemdash{} Broken Formatting}
  \begin{center}
    \inputminted[autogobble]{haskell}{snippets/outer-haskell-snippet-rule.hs_noformat}
  \end{center}
\end{frame}

\begin{frame}
  \frametitle{Snippet Rule \textemdash{} After Auto-Formatting}
  \begin{center}
    \inputminted[autogobble]{haskell}{snippets/haskell-snippet-rule.hs}
  \end{center}
\end{frame}

\begin{frame}
  \frametitle{Extracting Code}
  \begin{itemize}
  \item will always be up to date with the compiling source (yay)
  \item but we also have to format and maybe check again
  \end{itemize}
\end{frame}

\begin{frame}
  \frametitle{Checking Code}
  \begin{itemize}
  \item let's tackle checking first
  \item lots of times: broken code snippets that don't compile
  \item style errors you would notice in your actual setup
  \item after extracting a snippet into an includable file
  \item run linter/compiler/\ldots
  \item fail building presentation if the command fails
  \end{itemize}
\end{frame}

\begin{frame}
  \frametitle{Checking Code}
  \begin{itemize}
  \item haskell with hindent + hlint
  \item scala with sbt and scalafmt
  \item actually any programming language and linter
  \end{itemize}
\end{frame}

\begin{frame}
  \frametitle{Formatting Code}
  \begin{itemize}
  \item just another step like linting
  \item run formatter of choice on the source file
  \item e.g.\ format to a width of 55 chars
  \end{itemize}
\end{frame}

\begin{frame}
  \frametitle{Formatting Code}
  \begin{itemize}
  \item example of code:
  \end{itemize}
\end{frame}

\section{Pictures}

\begin{frame}
  \frametitle{Pictures}
  \begin{itemize}
  \item scenario 1: search on the web and download
    \begin{itemize}
    \item but you will forget from where
    \item resize and rotate are manual steps
    \item you have to store them in git
    \end{itemize}
  \item scenario 2: generated from description
    \begin{itemize}
    \item graphviz graphs
    \item ditaa diagrams
    \item plantuml diagrams
    \item and more\ldots{}
    \end{itemize}
  \end{itemize}
\end{frame}

\begin{frame}
  \frametitle{Downloading Pictures}
  \begin{itemize}
  \item use Haskell and Shake to download on demand
  \item download of the file from the internet
  \item file that describes from where plus transformations
  \item transformations performed by imagemagick
  \end{itemize}
\end{frame}

\begin{frame}
  \frametitle{Downloading Pictures}
  \begin{center}
    \includegraphics{images/maintain-make.jpg}
  \end{center}
\end{frame}

\begin{frame}
  \frametitle{Downloading Pictures}
    \begin{center}
    \inputminted{text}{images/maintain-make.src}
  \end{center}
\end{frame}

\begin{frame}
  \frametitle{Downloading Pictures}
    \begin{center}
    \inputminted{haskell}{snippets/download-images.hs}
  \end{center}
\end{frame}

\begin{frame}
  \frametitle{Generating Pictures}
  \begin{itemize}
  \item second scenario: picture is generated
  \item there is a file that describes it + tool to render
  \item Steps:
    \begin{itemize}
    \item write the description file
    \item generate graphic
    \item include in presentation
    \item change description
    \item generate graphic
    \item include in presentation
    \item change description again\ldots{}
    \end{itemize}
  \end{itemize}
\end{frame}

\begin{frame}
  \frametitle{Shake It}
  \begin{itemize}
  \item express the dependency as a shake rule
  \end{itemize}
  \inputminted{haskell}{snippets/graphviz-rule.hs}
\end{frame}

\begin{frame}
  \frametitle{Everything As A Rule}
  \begin{center}
    \includegraphics[width=\textwidth]{graphviz/rules-big.png}
  \end{center}
\end{frame}

\begin{frame}
  \frametitle{Getting Dependencies}
  \begin{itemize}
  \item the missing piece: how to ``discover'' dependencies?
  \item all of hackage is available
  \item parse LaTeX via HaTeX (this time)
  \item use the pandoc library
  \item \ldots whatever you need
  \end{itemize}
\end{frame}

\section{Continuous Integration}

\begin{frame}
  \frametitle{Develop Environment}
  \begin{itemize}
  \item we freely mixed stuff and used lots of tools
    \begin{itemize}
    \item haskell + libraries
    \item imagemagick
    \item graphviz
    \item ditaa
    \item LaTeX plus packages and special font
    \item scala, sbt, scalafmt
    \end{itemize}
  \end{itemize}
\end{frame}

\begin{frame}
  \frametitle{Continuous Integration via Travis}
  \inputminted[linenos=false, fontsize=\tiny, lastline=31]{yaml}{static-source/long-travis-ci.yml}
\end{frame}

\begin{frame}
  \frametitle{Continuous Integration via Travis}
  \inputminted[linenos=false, fontsize=\tiny, firstline=31, lastline=60]{yaml}{static-source/long-travis-ci.yml}
\end{frame}

\begin{frame}
  \frametitle{Continuous Integration via Travis}
  \inputminted[linenos=false, fontsize=\tiny, firstline=61, lastline=90]{yaml}{static-source/long-travis-ci.yml}
\end{frame}

\begin{frame}
  \frametitle{Continuous Integration via Travis}
  \inputminted[linenos=false, fontsize=\tiny, firstline=91, lastline=120]{yaml}{static-source/long-travis-ci.yml}
\end{frame}

\begin{frame}
  \frametitle{Continuous Integration Madness}
  \begin{itemize}
  \item it's huge and a mess, good luck maintaining this
  \item OS specific, your own setup vs travis
  \item not reproducible at all
  \item very brittle
  \end{itemize}
\end{frame}

\begin{frame}
  \frametitle{The One Command Lie}
  \begin{itemize}
  \item you just have to run this \textbf{one} command
  \item it's mostly a lie
  \item with nix, you can \textit{actually} achieve that!
  \item perfect: use it in ``.travis.yml'' as well as every pc
  \end{itemize}
\end{frame}

\begin{frame}
  \frametitle{Nix}
  \begin{itemize}
  \item \url{https://nixos.org/nix/}
  \end{itemize}
  \begin{quote}
    Nix is a powerful package manager for Linux and other Unix systems that makes package management reliable and reproducible.
  \end{quote}
\end{frame}

\begin{frame}
  \frametitle{Continuous Integration Made Easy}
  \inputminted{yaml}{snippets/travis.yml}
\end{frame}

\begin{frame}
  \frametitle{Executing Our Shake Build}
  \inputminted[breaklines]{yaml}{snippets/build-shebang.hs}
\end{frame}

\begin{frame}
  \frametitle{Only LaTeX?}
\end{frame}

\section{Conclusion}\label{sec:conclusion}

\begin{frame}
  \frametitle{Only LaTeX}
  \begin{itemize}
  \item all of this is not specific to LaTeX
  \item other: pandoc, reveal.js, \ldots{}
  \item e.g.\ download reveal.js automatically
  \item use pandoc to analyze the markdown
  \end{itemize}
\end{frame}

\begin{frame}
  \begin{center}
    \Huge
    Thanks for your attention
  \end{center}
  \begin{center}
    \Huge
    Markus Hauck (@markus1189)
  \end{center}
\end{frame}

\begin{frame}
  \tableofcontents{}
\end{frame}

\appendix{}

\section*{Bonus}\label{sec:bonus}

\end{document}
