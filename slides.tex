\documentclass{beamer}

% Must be loaded first
\usepackage{tikz}

\usepackage[utf8]{inputenc}
\usepackage{textpos}

% Font configuration
\usepackage{fontspec}

\input{font.tex}

% Tikz for beautiful drawings
\usetikzlibrary{mindmap,backgrounds}
\usetikzlibrary{arrows.meta,arrows}
\usetikzlibrary{shapes.geometric}

% Minted configuration for source code highlighting
\usepackage{minted}
\setminted{highlightcolor=black!5, linenos}
\setminted{style=lovelace}

\usepackage[listings, minted]{tcolorbox}
\tcbset{left=6mm}

% Use the include theme
\usetheme{codecentric}

% Metadata
\title{How This Presentation Was Made}
\author{Markus Hauck @markus1189}

\newcommand{\recipe}{%
  \begin{itemize}
  \item AST
  \item \texttt{inject}
  \item interpreter
  \item check laws
  \end{itemize}
}

% The presentation content
\begin{document}

\begin{frame}[noframenumbering,plain]
  \titlepage{}
\end{frame}

\section{Introduction}\label{sec:introduction}

\begin{frame}
  \frametitle{How I Write Presentations}
  \begin{itemize}
  \item Most of the time: LaTeX Beamer
  \item very nice pdf slides
  \item you can use LaTeX packages
  \end{itemize}
\end{frame}

\begin{frame}
  \frametitle{Pain Points}
  \begin{itemize}
  \item you have to setup LaTeX
  \item plus all required packages (most of the time: \textbf{all})
  \item hope that everything is up to date and try to build it
  \item notice missing fonts, try to find them somewhere
  \item notice missing python package (pygments), install it
  \item lots of trial and error \texttt{:(}
  \end{itemize}
\end{frame}

\begin{frame}
  \frametitle{Presentations}
  \begin{itemize}
  \item having CI would be nice :thinking:
  \item just check out my github and build it
  \item 100 lines of .travis.yml, specific to Ubuntu
  \item good luck with that!
  \end{itemize}
\end{frame}

\begin{frame}
  \frametitle{Preview}
  \begin{itemize}
  \item look at common scenarios
  \item present a solution
  \end{itemize}
\end{frame}

\begin{frame}
  \frametitle{Scenarios}
  \begin{itemize}
  \item Nix, Shake and Dhall
  \item Scenario 1: Dependencies
  \item Scenario 2a: Editing Code
  \item Scenario 2b: Checking Snippets
  \item Scenario 3: Editing Pictures
  \item Scenario 4: Continuous Integration
  \end{itemize}
\end{frame}

\begin{frame}
  \frametitle{Editing Code}
  \begin{itemize}
  \item Step 1: Implement your code in a Scala project
  \item Step 2: Wild Copy And Paste Into Presentation
  \item Step 3: Reformat To Fit Slide
  \item Step 4: Change Original Source Code
  \item Step 5: Wild Editing Of Code on Slides
  \item Step 6: Being suspicious that something is broken (optional)
  \end{itemize}
\end{frame}

\begin{frame}
  \frametitle{Editing Code}
  \begin{itemize}
  \item we need a better way
  \item idea: extract source code directly from actual project
  \item use comments to delimit ``snippets''
  \item write code to extract everything in between
  \end{itemize}
\end{frame}

\begin{frame}
  \frametitle{Editing Code}
  \begin{itemize}
  \item add comments in the code
  \item write a small ``snippet'' file
  \item let shake automatically extract snippets
  \item include code snippets in presentation
  \end{itemize}
\end{frame}

\begin{frame}
  \frametitle{Editing Code}
  \begin{itemize}
  \item okay that gets rid of the copy + pasting
  \item what about formatting and checking?
  \item let's tackle checking first
  \item lots of times: broken code snippets that never compile
  \item mostly hand waving examples
  \item style errors you would notice in your actual setup
  \item wouldn't it be cool to get that for your presentation
  \end{itemize}
\end{frame}

\begin{frame}
  \frametitle{Checking Code}
  \begin{itemize}
  \item after extracting a snippet into an includable file
  \item run linter/compiler/...
  \item fail building presentation if the command fails
  \end{itemize}
\end{frame}

\begin{frame}
  \frametitle{Formatting Code}
  \begin{itemize}
  \item just another step like linting
  \item run formatter of choice on the source file
  \item e.g. format to a width of 55 chars
  \end{itemize}
\end{frame}

\begin{frame}
  \frametitle{Formatting Code}
  \begin{itemize}
  \item example of code:
  \end{itemize}
\end{frame}

\begin{frame}
  \frametitle{Pictures}
  \begin{itemize}
  \item need more pictures
  \item downloaded, but from where?
  \item maybe needs conversion and scaling
  \item commit them to the repository?
  \item generated by an external tool (ditaa, plantuml, graphviz, ...)
  \item we'll tackle both
  \end{itemize}
\end{frame}

\begin{frame}
  \frametitle{Downloading Pictures}
  \begin{itemize}
  \item write a new Shake rule of course
  \item normal download of the file from the internet
  \item provide a file per image that contains url + imagemagick instructions
  \item the rule in Haskell: code
  \end{itemize}
\end{frame}

\begin{frame}
  \frametitle{Generating Pictures}
  \begin{itemize}
  \item common scenario: picture is generated
  \item there is a file that describes it + tool to render
  \item Steps:
    \begin{itemize}
    \item write the description file
    \item generate graphic
    \item include in presentation
    \item notice change needed :(
    \end{itemize}
  \end{itemize}
\end{frame}

\begin{frame}
  \frametitle{The One Command Lie}
  \begin{itemize}
  \item you just have to run this \textbf{one} command
  \item it's mostly a lie
  \item with nix, you can \textit{actually} achieve that!
  \item perfect: use it in .travis.yml as well as every pc
  \end{itemize}
\end{frame}



\section{Conclusion}\label{sec:conclusion}

\begin{frame}
  \begin{center}
    \Huge
    Thanks for your attention
  \end{center}
  \begin{center}
    \Huge
    Markus Hauck (@markus1189)
  \end{center}
\end{frame}

\begin{frame}
  \tableofcontents{}
\end{frame}

\appendix{}

\section*{Bonus}\label{sec:bonus}

\end{document}
